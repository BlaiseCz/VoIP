\documentclass[12pt,a4paper]{article}
\usepackage[polish]{babel}
\usepackage[T1]{fontenc}
\usepackage[utf8x]{inputenc}
\usepackage{hyperref}
\usepackage{url}
\usepackage{graphicx}
\usepackage{float}

\usepackage{color}
%\newcommand{\todo}[1]{\\ \textcolor{red}{TODO: #1}\PackageWarning{TODO:}{#1!}}
\newcommand{\todo}[1]{\\ \textcolor{red}{TODO: #1}}

\addtolength{\hoffset}{-1.5cm}
\addtolength{\marginparwidth}{-1.5cm}
\addtolength{\textwidth}{3cm}
\addtolength{\voffset}{-1cm}
\addtolength{\textheight}{2.5cm}
\setlength{\topmargin}{0cm}
\setlength{\headheight}{0cm}

\begin{document}

    \title{Opracowanie bezpiecznego systemu komunikacji głosowej
    w sieci IP (VoIP) wraz z jego implementacją.
        \\ \\Telefonia IP}

    \author{Krzysztof Czarnecki 136224\\Piotr Zieliński 136333}
    \date{\today}
    \maketitle

    \newpage
    \tableofcontents

    \section{Charakterystyka ogólna projektu}
    \section{Architektura systemu}

Aplikacja wykonana będzie w architekturze klient-serwer, gdzie serwerem będzie aplikacja obsługująca realizację
połączeń oraz przechowywała historię połączeń.
Aplikacje klienckie będą służyły do wykonywania połączeń.
\\
\\
\textbf{Aplikacja kliencka} będzie się składać z maksymalnie 4 modułów:
\begin{itemize}
	\item Moduł obsługi mikrofonu,
	\item Moduł obsługi wyjścia audio,
	\item Moduł komunikacji przychodzącej,
	\item Moduł komunikacji wychodzącej.
\end{itemize}
\
\textbf{Aplikacja serwera} będzie składać się z:
\begin{itemize}
	\item Modułu nawiązywania sesji,
	\item Modułu obsługi trweających sesji.
\end{itemize}
    \section{Wymagania}

\subsection{Funkcjonalne}

\begin{itemize}
    \item Uwierzytelnianie
    \item Nawiązanie połączenia
    \item Odebranie/odrzucenie połączenia
    \item Opuszczenie rozmowy - rozłączenie się
\end{itemize}


\subsection{Niefunkcjonalne}

\begin{itemize}
    \item Ustalanie wspólnego klucza tajnego
    \item Szyfrowanie rozmowy
    \item Komunikacja sieciowa pomiędzy użytkownikami
\end{itemize}
    \section{Narzędzia}
\end{document}

% PUNKTORY
% \begin{itemize}
% \item Public Documents
% \item All Documents
% \item Create Document
% \end{itemize}

% obrazek
% \begin{figure}[H]
% \centering
% \includegraphics[width=15cm]{pictures/deszyf_mycbc.png}
% \caption{Wykres zależności czasu deszyfrowania [ms] od wielkości pliku [MB] z uwzględnieniem własnej implementacji szyfru CBC}
% \label{pictures/szyfrowanie.png}
% \end{figure}